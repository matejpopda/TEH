\documentclass[../main.tex]{subfiles}
\graphicspath{{\subfix{../images/}}}


% \newcommand*{\cross}{\times} %defined in main.tex

\begin{document}





\subsection*{1 A}


\begin{quote}
    Popište vyhrávající strategii prvního hráče ve hře $3 \cross 3 \cross 3$ piškvorky.
\end{quote}


V prvním tahu jako první hráč zahrajeme $\cross$ na pozici $[2,2,2]$, tedy uprostřed.

Druhý hráč jako odpověď může zahrát jeden z následujících 3 tahů.
\begin{enumerate}
    \item Zahraje $\circ$ do rohu, BÚNO $[1,1,1]$
    \item Zahraje $\circ$ na hranu, BÚNO $[2,1,1]$
    \item Zahraje $\circ$ do strany, BÚNO $[2,2,1]$
\end{enumerate}
Toto je bez úhony na obecnosti, protože hra je symetrická pro otáčení a zrcadlení.

První hráč nezávisle na tahu druhého hráče následně zahraje tah $\cross$ na pozici $[1,1,2]$.

Druhý hráč je nucen zahrát $\circ$ na $[3,3,2]$ (pokud tak neučiní zahraje na toto pole první hráč ve svém tahu a vyhraje.)

První hráč následovně zahraje $\cross$ na pozici $[1,2,2]$. 

Druhý hráč je nyní nucen zahrát $\circ$ na 2 pozice, pozici $[1,3,2]$ a pozici $[3,2,2]$. Toto je ale nemožné. 
První hráč na svém kole tedy buded určitě schopen zahrát jednu z těchto pozic a vyhrát.

\subsection*{1 B}

\begin{quote}
    Zkonstruujte 2-obarvení $3 \times 3 \times 3$, 
    které nemá žádnou monochromatickou kombinatorickou přímku.
    
\end{quote}

Máme abecedu $A = \{ 1,2,3 \}$ a dimenzi $d = 3$. 

Obarvení označme jako $\{X, O\}$.

Máme následující vzory:

\begin{multicols}{4}
\begin{itemize}
    \item $\tau_1    = * * *$
    \item $\tau_2    = 1 * *$
    \item $\tau_3    = 2 * *$
    \item $\tau_4    = 3 * *$
    \item $\tau_5    = * 1 *$
    \item $\tau_6    = * 2 *$
    \item $\tau_7    = * 3 *$
    \item $\tau_8    = * * 1$
    \item $\tau_9    = * * 2$
    \item $\tau_0 = * * 3$
    \item $\tau_{10} = 1 1 *$
    \item $\tau_a = 1 2 *$
    \item $\tau_b = 1 3 *$
    \item $\tau_c = 1 * 1$
    \item $\tau_d = 1 * 2$
    \item $\tau_e = 1 * 3$
    \item $\tau_f = 2 1 *$
    \item $\tau_g = 2 2 *$
    \item $\tau_h = 2 3 *$
    \item $\tau_i = 2 * 1$
    \item $\tau_j = 2 * 2$
    \item $\tau_k = 2 * 3$
    \item $\tau_l = 3 1 *$
    \item $\tau_m = 3 2 *$
    \item $\tau_n = 3 3 *$
    \item $\tau_o = 3 * 1$
    \item $\tau_p = 3 * 2$
    \item $\tau_q = 3 * 3$
    \item $\tau_r = * 1 1$
    \item $\tau_s = * 1 2$
    \item $\tau_t = * 1 3$
    \item $\tau_u = * 2 1$
    \item $\tau_v = * 2 2$
    \item $\tau_w = * 2 3$
    \item $\tau_y = * 3 1$
    \item $\tau_x = * 3 2$
    \item $\tau_z = * 3 3$
\end{itemize}
\end{multicols}

Máme tedy 37 kombinatorických přímek které nemají být monochromatické.


Postupujme následovně.

Každé slovo obarvíme jako $X$. Toto obarvení zřejmě není monochromatické.

Změňme obarvení následujících slov: $311,221, 131$ na $O$. 
Tyto slova mohou mít stejnou barvu, protože netvoří kombinatorickou přímku. 
Jedná se ale o propojení protějších rohů krychle. 

To samé provedeme pro i pro slova $122, 212, 332$ a $113, 323, 233$.

Pokud nějaký vzor obsahuje jedno z předchozích slov, jeho kombinatorická přímka přestává být monochromatickou.

Jediný vzor který zbyde a je monochromatický je vzor $\tau_1 = * * *$, 
ten ale přestane být monochromatický poté co  
obarvíme $111$ na $X$. Můžeme jednoduše zkontrolovat ,že obarvení tohoto slova
nevytvoří žádnou novou monochromatickou přímku. 



Tímto jsme zkonstruovali obarvení které neobsahuje žádnou monochromatickou přímku. 










\end{document}