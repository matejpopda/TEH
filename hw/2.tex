\documentclass[../main.tex]{subfiles}
\graphicspath{{\subfix{../images/}}}
\begin{document}



\section*{Zadání}

Pro konečnou množinu $S$ kladných celých čísel definujeme $S$-sirkovou nestrannou 
normální hru tak, že hráč na tahu si vybere nějaké $s \in S$ a odebere z hrací plochy přesně $s$ sirek. 

\section*{1a}

Buď $\alpha_n$ pozice v $\{1,4,5\}$-sirkové hře s $n$ sirkami. Pro která $n\in \mathbb{N}$ je $\alpha_n$ typu N? 


Budeme používat následující tvrzení z přednášky: 
\begin{enumerate}
    \item Pozice je typu P, pokud všechny tahy z ní vedou do pozice typu N
    \item Pozice je typu N, pokud existuje tah do pozice typu P 
\end{enumerate}


Vypišme několik prvních pozic.

\begin{center}
    \begin{tabular}{|c|c|c|}
        \hline
        Pozice & Typ & Možné tahy \\
        \hline
        $\alpha_0$ & P & $\emptyset$ \\
        \hline
        $\alpha_1$ & N & $\{ \alpha_0 \}$ \\
        \hline
        $\alpha_2$ & P & $\{ \alpha_1 \}$ \\
        \hline
        $\alpha_3$ & N & $\{ \alpha_2 \}$ \\
        \hline
        $\alpha_4$ & N & $\{ \alpha_3, \alpha_0 \}$ \\
        \hline
        $\alpha_5$ & N & $\{ \alpha_4,  \alpha_1, \alpha_0 \}$ \\
        \hline
        $\alpha_6$ & N & $\{ \alpha_5,  \alpha_2, \alpha_1 \}$ \\
        \hline
        $\alpha_7$ & N & $\{ \alpha_6,  \alpha_3, \alpha_2 \}$ \\
        \hline
        $\alpha_8$ & P & $\{ \alpha_7,  \alpha_4, \alpha_3 \}$ \\
        \hline
        $\alpha_9$ & N & $\{ \alpha_8,  \alpha_5, \alpha_4 \}$ \\
        \hline
        $\alpha_{10}$ & P & $\{ \alpha_9,  \alpha_6, \alpha_5 \}$ \\
        \hline
        $\alpha_{11}$ & N & $\{ \alpha_{10},  \alpha_7, \alpha_6 \}$ \\
        \hline
        $\alpha_{12}$ & N & $\{ \alpha_{11},  \alpha_8, \alpha_7 \}$ \\
        \hline
        $\alpha_{13}$ & N & $\{ \alpha_{12},  \alpha_9, \alpha_8 \}$ \\
        \hline
        $\alpha_{14}$ & N & $\{ \alpha_{13},  \alpha_{10}, \alpha_9 \}$ \\
        \hline
        $\alpha_{15}$ & N & $\{ \alpha_{14},  \alpha_{11}, \alpha_{10} \}$ \\
        \hline
        $\alpha_{16}$ & P & $\{ \alpha_{15},  \alpha_{12}, \alpha_{11} \}$ \\
        \hline
        $\alpha_{17}$ & N & $\{ \alpha_{16},  \alpha_{13}, \alpha_{12} \}$ \\
        \hline
        $\alpha_{18}$ & P & $\{ \alpha_{17},  \alpha_{14}, \alpha_{13} \}$ \\
        \hline
        $\alpha_{19}$ & N & $\{ \alpha_{18},  \alpha_{15}, \alpha_{14} \}$ \\
        \hline
    \end{tabular}

\end{center}

Vidíme, že pro pozice $\alpha_{8k}$ a $\alpha_{2+8k}, k\in\mathbb{N}_0$ se jedná o pozici P. 

Dokážeme indukcí přes k.

Případ pro $k=0,1$ je v tabulce. 

Indukční předpoklad: 
Tvrzení platí pro $j<k$.

Důkaz. Mějme nějaké pevné k. Můžeme si vytvořit tabulku pro tahy.

\begin{center}
    \begin{tabular}{|c|c|}
        \hline
        Pozice & Možné tahy \\
        \hline
        $\alpha_{8k}$  & $\{ \alpha_{8k-1},  \alpha_{8k-4}, \alpha_{8k-5} \}$ \\
        \hline
        $\alpha_{8k+1}$  & $\{ \alpha_{8k},  \alpha_{8k-3}, \alpha_{8k-4} \}$ \\
        \hline
        $\alpha_{8k+2}$  & $\{ \alpha_{8k+1},  \alpha_{8k-2}, \alpha_{8k-3} \}$ \\
        \hline
        $\alpha_{8k+3}$  & $\{ \alpha_{8k+2},  \alpha_{8k-1}, \alpha_{8k-2} \}$ \\
        \hline
        $\alpha_{8k+4}$  & $\{ \alpha_{8k+3},  \alpha_{8k}, \alpha_{8k-1} \}$ \\
        \hline
        $\alpha_{8k+5}$  & $\{ \alpha_{8k+4},  \alpha_{8k+1}, \alpha_{8k} \}$ \\
        \hline
        $\alpha_{8k+6}$  & $\{ \alpha_{8k+5},  \alpha_{8k+2}, \alpha_{8k+1} \}$ \\
        \hline
        $\alpha_{8k+7}$ & $\{ \alpha_{8k+6},  \alpha_{8k+3}, \alpha_{8k+2} \}$ \\
        \hline
    \end{tabular}

\end{center}


Přeindexujeme některé tahy a dostaneme


\begin{center}
    \begin{tabular}{|c|c|}
        \hline
        Pozice &  Možné tahy \\
        \hline
        $\alpha_{8k}$ & $\{ \alpha_{8(k-1)+7},  \alpha_{8(k-1)+4}, \alpha_{8(k-1)+3} \}$ \\
        \hline
        $\alpha_{8k+1}$  & $\{ \alpha_{8k},  \alpha_{8(k-1)+5}, \alpha_{8(k-1)+4} \}$ \\
        \hline
        $\alpha_{8k+2}$ & $\{ \alpha_{8k+1},  \alpha_{8(k-1)+6}, \alpha_{8(k-1)+5} \}$ \\
        \hline
        $\alpha_{8k+3}$  & $\{ \alpha_{8k+2},  \alpha_{8(k-1)+7}, \alpha_{8(k-1)+6} \}$ \\
        \hline
        $\alpha_{8k+4}$  & $\{ \alpha_{8k+3},  \alpha_{8k}, \alpha_{8(k-1)+7} \}$ \\
        \hline
        $\alpha_{8k+5}$  & $\{ \alpha_{8k+4},  \alpha_{8k+1}, \alpha_{8k} \}$ \\
        \hline
        $\alpha_{8k+6}$  & $\{ \alpha_{8k+5},  \alpha_{8k+2}, \alpha_{8k+1} \}$ \\
        \hline
        $\alpha_{8k+7}$ & $\{ \alpha_{8k+6},  \alpha_{8k+3}, \alpha_{8k+2} \}$ \\
        \hline
    \end{tabular}

\end{center}


Konečně využijeme IP a doplníme typ tahů.

\begin{center}
    \begin{tabular}{|c|c|c|}
        \hline
        Pozice & Typ & Možné tahy \\
        \hline
        $\alpha_{8k}$ & P & $\{ \alpha_{8(k-1)+7},  \alpha_{8(k-1)+4}, \alpha_{8(k-1)+3} \}$ \\
        \hline
        $\alpha_{8k+1}$ & N & $\{ \alpha_{8k},  \alpha_{8(k-1)+5}, \alpha_{8(k-1)+4} \}$ \\
        \hline
        $\alpha_{8k+2}$ & P & $\{ \alpha_{8k+1},  \alpha_{8(k-1)+6}, \alpha_{8(k-1)+5} \}$ \\
        \hline
        $\alpha_{8k+3}$ & N & $\{ \alpha_{8k+2},  \alpha_{8(k-1)+7}, \alpha_{8(k-1)+6} \}$ \\
        \hline
        $\alpha_{8k+4}$ & N & $\{ \alpha_{8k+3},  \alpha_{8k}, \alpha_{8(k-1)+7} \}$ \\
        \hline
        $\alpha_{8k+5}$ & N & $\{ \alpha_{8k+4},  \alpha_{8k+1}, \alpha_{8k} \}$ \\
        \hline
        $\alpha_{8k+6}$ & N & $\{ \alpha_{8k+5},  \alpha_{8k+2}, \alpha_{8k+1} \}$ \\
        \hline
        $\alpha_{8k+7}$ & N & $\{ \alpha_{8k+6},  \alpha_{8k+3}, \alpha_{8k+2} \}$ \\
        \hline
    \end{tabular}

\end{center}


Což je to co jsme chtěli ukázat. 


\section*{1b}


Pro všechna $n \in \mathbb{N}$ najděte $a_n$ t.ž. $\alpha_n \equiv \star a_n$.


Opět si napišme tabulku. Pro její vyplnění používáme MEX princip. 

\begin{center}
    \begin{tabular}{|c|c|c|}
        \hline
        Pozice $\alpha_i$ & $\star a_i$ & Možné tahy \\
        \hline
        $\alpha_0$ & 0 & $\emptyset$ \\
        \hline
        $\alpha_1$ & 1 & $\{ \alpha_0 \}$ \\
        \hline
        $\alpha_2$ & 0 & $\{ \alpha_1 \}$ \\
        \hline
        $\alpha_3$ & 1 & $\{ \alpha_2 \}$ \\
        \hline
        $\alpha_4$ & 2 & $\{ \alpha_3, \alpha_0 \}$ \\
        \hline
        $\alpha_5$ & 3 & $\{ \alpha_4,  \alpha_1, \alpha_0 \}$ \\
        \hline
        $\alpha_6$ & 2 & $\{ \alpha_5,  \alpha_2, \alpha_1 \}$ \\
        \hline
        $\alpha_7$ & 3 & $\{ \alpha_6,  \alpha_3, \alpha_2 \}$ \\
        \hline
        $\alpha_8$ & 0 & $\{ \alpha_7,  \alpha_4, \alpha_3 \}$ \\
        \hline
        $\alpha_9$ & 1 & $\{ \alpha_8,  \alpha_5, \alpha_4 \}$ \\
        \hline
        $\alpha_{10}$ & 0 & $\{ \alpha_9,  \alpha_6, \alpha_5 \}$ \\
        \hline
        $\alpha_{11}$ & 1 & $\{ \alpha_{10},  \alpha_7, \alpha_6 \}$ \\
        \hline
        $\alpha_{12}$ & 2 & $\{ \alpha_{11},  \alpha_8, \alpha_7 \}$ \\
        \hline
        $\alpha_{13}$ & 3 & $\{ \alpha_{12},  \alpha_9, \alpha_8 \}$ \\
        \hline
        $\alpha_{14}$ & 2 & $\{ \alpha_{13},  \alpha_{10}, \alpha_9 \}$ \\
        \hline
        $\alpha_{15}$ & 3 & $\{ \alpha_{14},  \alpha_{11}, \alpha_{10} \}$ \\
        \hline
        $\alpha_{16}$ & 0 & $\{ \alpha_{15},  \alpha_{12}, \alpha_{11} \}$ \\
        \hline
        $\alpha_{17}$ & 1 & $\{ \alpha_{16},  \alpha_{13}, \alpha_{12} \}$ \\
        \hline
        $\alpha_{18}$ & 0 & $\{ \alpha_{17},  \alpha_{14}, \alpha_{13} \}$ \\
        \hline
        $\alpha_{19}$ & 1 & $\{ \alpha_{18},  \alpha_{15}, \alpha_{14} \}$ \\
        \hline
    \end{tabular}

\end{center}

Všimněme si že $a_i$ se opakuje vždy po 8 krocích. Dokážeme úplně stejně jako první úlohu, indukcí.

Dokážeme indukcí přes k.

Případ pro $k=0,1$ je v tabulce. 

Indukční předpoklad: 
Tvrzení platí pro $j<k$. Platí tedy následující: 
\begin{center}
    \begin{tabular}{|c|c|}
        \hline
        Pozice $\alpha_i$ & $\star a_i$\\
        \hline
        $\alpha_{(8k-1)+0}$ & 0  \\
        \hline
        $\alpha_{(8k-1)+1}$ & 1 \\
        \hline
        $\alpha_{(8k-1)+2}$ & 0  \\
        \hline
        $\alpha_{(8k-1)+3}$ & 1 \\
        \hline
        $\alpha_{(8k-1)+4}$ & 2  \\
        \hline
        $\alpha_{(8k-1)+5}$ & 3 \\
        \hline
        $\alpha_{(8k-1)+6}$ & 2  \\
        \hline
        $\alpha_{(8k-1)+7}$ & 3 \\
        \hline
    \end{tabular}

\end{center}


Důkaz. Mějme nějaké pevné k. Chceme dokázat že platí následující tabulka tahů. Rovnou jsme ji přeindexovali.

\begin{center}
    \begin{tabular}{|c|c|}
        \hline
        Pozice &  Možné tahy \\
        \hline
        $\alpha_{8k}$ & $\{ \alpha_{8(k-1)+7},  \alpha_{8(k-1)+4}, \alpha_{8(k-1)+3} \}$ \\
        \hline
        $\alpha_{8k+1}$  & $\{ \alpha_{8k},  \alpha_{8(k-1)+5}, \alpha_{8(k-1)+4} \}$ \\
        \hline
        $\alpha_{8k+2}$ & $\{ \alpha_{8k+1},  \alpha_{8(k-1)+6}, \alpha_{8(k-1)+5} \}$ \\
        \hline
        $\alpha_{8k+3}$  & $\{ \alpha_{8k+2},  \alpha_{8(k-1)+7}, \alpha_{8(k-1)+6} \}$ \\
        \hline
        $\alpha_{8k+4}$  & $\{ \alpha_{8k+3},  \alpha_{8k}, \alpha_{8(k-1)+7} \}$ \\
        \hline
        $\alpha_{8k+5}$  & $\{ \alpha_{8k+4},  \alpha_{8k+1}, \alpha_{8k} \}$ \\
        \hline
        $\alpha_{8k+6}$  & $\{ \alpha_{8k+5},  \alpha_{8k+2}, \alpha_{8k+1} \}$ \\
        \hline
        $\alpha_{8k+7}$ & $\{ \alpha_{8k+6},  \alpha_{8k+3}, \alpha_{8k+2} \}$ \\
        \hline
    \end{tabular}

\end{center}

Využijeme IP a princip MEX a dostaneme

\begin{center}
    \begin{tabular}{|c|c|c|}
        \hline
        Pozice $\alpha_i$ & $\star a_i$ & Možné tahy \\
        \hline
        $\alpha_{8k}$ & 0 & $\{ \alpha_{8(k-1)+7},  \alpha_{8(k-1)+4}, \alpha_{8(k-1)+3} \}$ \\
        \hline
        $\alpha_{8k+1}$ & 1 & $\{ \alpha_{8k},  \alpha_{8(k-1)+5}, \alpha_{8(k-1)+4} \}$ \\
        \hline
        $\alpha_{8k+2}$ & 0 & $\{ \alpha_{8k+1},  \alpha_{8(k-1)+6}, \alpha_{8(k-1)+5} \}$ \\
        \hline
        $\alpha_{8k+3}$ & 1 & $\{ \alpha_{8k+2},  \alpha_{8(k-1)+7}, \alpha_{8(k-1)+6} \}$ \\
        \hline
        $\alpha_{8k+4}$ & 2 & $\{ \alpha_{8k+3},  \alpha_{8k}, \alpha_{8(k-1)+7} \}$ \\
        \hline
        $\alpha_{8k+5}$ & 3 & $\{ \alpha_{8k+4},  \alpha_{8k+1}, \alpha_{8k} \}$ \\
        \hline
        $\alpha_{8k+6}$ & 2 & $\{ \alpha_{8k+5},  \alpha_{8k+2}, \alpha_{8k+1} \}$ \\
        \hline
        $\alpha_{8k+7}$ & 3 & $\{ \alpha_{8k+6},  \alpha_{8k+3}, \alpha_{8k+2} \}$ \\
        \hline
    \end{tabular}

\end{center}


Což je to co jsme chtěli dokázat. 




\end{document}